In the following we will use $\varphi/\psi$ to denote node ansatz functions with linear/quadratic master basis functions, respectively.
We define the spaces
\begin{align*}
	\H^1 &:= \langle \psi_1,\ldots,\psi_{N_P} \rangle \subset C^0(\Or)\\
	\H^2 &:= \langle \varphi_1,\ldots,\varphi_N \rangle \subset C^1(\Or)
\end{align*} 
For discretization we will use Taylor-Hood elements, which means $\chi\in\H^2, p\in\H^1$ via
\begin{align}
	\chi(X,t) &= \sumi \vc_i(t)\varphi_i(X), && \vc_i(t):[0,T] \to\R^3.\label{def:chi}\\
	p(X,t) &= \sumiP d_i(t)\psi_i(X), && d_i(t):[0,T] \to\R.
\end{align}
This means we have quadratic ansatz functions for positions, velocity and linear functions for pressure.
With \eqref{def:chi} we deduce the discrete quantities
\begin{align*}
	\vV(X,t) &= \d{\chi}{t}(X,t) = \sumi \vc_i'(t)\varphi_i(X)\\
	\vA(X,t) &= \d{\vV}{t}(X,t) = \sumi \vc_i''(t)\varphi_i(X)\\
	\vF(X,t) &= \d{\chi}{X}(X,t) = \m{\grad \chi_1(X,t)\\ \grad \chi_2(X,t)\\ \grad \chi_3(X,t)}
		   = \sumi\m{c_{i,1}\d{\varphi_i}{X_1} & \dots & c_{i,1}\d{\varphi_i}{X_3}\\
		   				\vdots & \ddots & \vdots\\
		   				c_{i,3}\d{\varphi_i}{X_1} & \dots & c_{i,3}\d{\varphi_i}{X_3}}\\
		   &=\sumi \vc_i(t)\cdot \grad\varphi_i(X) = \sumi \vc_i(t)\otimes \dpi,\\
	\dot{\vF}(X,t) &= \sumi \vc'_i(t)\otimes \dpi.
\end{align*}

Now, equation \eqref{def:maineq} and the incompressibility constraint give the system 
\begin{align}
	\rho_0(X)\d{\vV}{t}(X,t) &= \divergence\vP(X,t) + \vB(X,t)&& \fo X\in\Or,\label{def:mainsys1}\\
	J(X,t) &= 1 && \fo X\in\Or,\label{def:maineq_cond}
\end{align}
for which we have the spatial weak form
\begin{align}
	\intor\rho_0(X)\d{\vV}{t}(X,t)\varphi(X)dX &= \intor\divergence\vP(X,t)\varphi(X)dX + \intor\vB(X,t)\varphi(X)dX &&\fo \varphi\in\H^2\label{def:mainsys1_weak}\\
	\intor (J(X,t)-1)\psi(X)dX &= 0 && \fo \psi\in\H^1\label{def:maineq_cond_weak}
\end{align}
The first summand on the right hand side of \eqref{def:mainsys1_weak} can be transformed (via integration by parts and Gauss' Theorem) as
\begin{align*}
	\intor \div{\vP(X,t)}\varphi(X) dX &= \intor \div{\vP(X,t)\varphi(X)} - \vP(X,t)\divergence\varphi(X) dX\\
		 &= \intorb \vP(X,t)\varphi(X)\vec{N}d\vec{N} - \intor \vP(X,t)\divergence\varphi(X)dX.
\end{align*}
Now, to satisfy equations \eqref{def:mainsys1_weak} and \eqref{def:maineq_cond_weak}, they must hold true for every node basis function. 
Consequently, we obtain a system of $3N+N_P$ equations
\begin{align}
	&\intor \rho_0(X)\d{\vV}{t}(X,t)\varphi_k(X)dX + \intor\vP(X,t)\dpk dX\label{def:mainsys1_weak_kparts}\\
	&= \intorb \vP(X,t)\varphi_k(X)\vec{N}d\vec{N} + \intor\vB(X,t)\varphi_k(X)dX, && \fo k=1\ldots N,\nonumber\\
	0 &= \intor (J(X,t)-1)\psi_k(X)dX, && \fo k=1\ldots N_P.\label{def:maineq_cond_weak_kparts}
\end{align}

\subsubsection{Integrals of main equation parts}
In the following we fix a $k\in\{1,\ldots, N\}$.
Then, by \eqref{eq:fphidx} we obtain
\begin{align}
	\intor \rho_0(X)\d{\vV}{t}(X,t)\varphi_k(X)dX\nonumber
		&\re{eq:fphidx} \sumvk\sumgp w_p \rho_0(\pmp)\d{\vV}{t}(\pmp,t)\Nkmp\jmp\nonumber\\
		&= \sumvk\sumgp w_p \rho_0(\pmp)\sumi \vc_i''(t)\varphi_i(\pmp) \Nkmp\jmp\nonumber\\
		&= \sumi \vc_i''(t)\sumvk\sumgp w_p \rho_0(\pmp)N_{l(i,m)}(X_p) \Nkmp\jmp,\label{def:discreteVphidX}\\
	\intor\vB(X,t)\varphi_k(X)dX &= \sumvk\sumgp w_p \vB(\pmp,t)\Nkmp\jmp\nonumber.
\end{align}
Similarly, the second summand of \eqref{def:mainsys1_weak_kparts} writes as
\begin{align}
		\intor\vP(X,t)\dpk dX &\re{eq:fgradphidx} \sumvk\sumgp w_p\vP(\pmp,t)\dNkmp\jmp.\label{def:discretePgradphidX}
\end{align}
Finally, for $k\in\{1\ldots N_P\}$, condition \eqref{def:maineq_cond} formulates to
\begin{align}
	\intor (J(X,t)-1)\psi_k(X)dX &= \sumvk\sumgp w_p (\det\vF(\pmp,t)-1) \Nkmp\jmp\label{eq:weak_incomp_condition}\\
	&= \sumvk\sumgp w_p \left(\det\left(\sumi \vc_i(t)\otimes \divergence\varphi_i(\pmp)\right)-1\right) \Nkmp\jmp.\nonumber
\end{align}

\subsubsection{Local tensor evaluations}
The most elaborative part is to evaluate the stress tensor $\vP$ at all volumes and Gauss points.
We have from \eqref{def:completeP} that
\begin{align*}
		\vP(\pmp,t) &= p(\pmp,t)\vF^{-T}(\pmp,t) + 2(c_{10} + I_1(\vC(\pmp,t))c_{01})\vF(\pmp,t)\\
			 &- 2c_{01}\vF(\pmp,t)\vC(\pmp,t)+g(\la(\pmp,t))\vF(\pmp,t)\va_0(\pmp)\otimes\va_0(\pmp)\\
			 & + \eta \dot{\vF}(\pmp,t).
\end{align*}
In the following we derive representations of the various components of $\vP$, where we assume $X\in\Or$.
\begin{align*}
	\vF(\pmp,t) &= \sumi \vc_i(t)\otimes \dNimp,\\
	\dot{\vF}(\pmp,t) &= \sumi \vc_i'(t)\otimes \dNimp,
\end{align*}	
\begin{align}
	I_1(\vC(X,t)) &= \tr\vC(X,t) = \tr(\vF(X,t)^T\vF(X,t)) = \vF(X,t) : \vF(X,t)\nonumber\\
	&= \sumi \vc_i(t)\otimes \dpi : \sumi \vc_i(t)\otimes \dpi\nonumber\\
	&= \sumi\sumj \left(\vc_i(t)\otimes \dpi\right) : (\vc_j(t)\otimes \dpj)\nonumber\\
	&= \sumi\sumj (\vc_i(t) \cdot \vc_j(t))(\dpi \cdot \dpj),\label{eq:I1_basis}\\
	\Rightarrow I_1(\vC(\pmp,t)) &= \sumi\sumj (\vc_i(t) \cdot \vc_j(t))(\dNimp \cdot \dNjmp),\nonumber
\end{align}
Expression for fibre stretch $\la$:
\begin{align*}
% 	\vF(X,t)^T\vF(X,t) &= \sumi\sumj \bigl(\vc_i(t)\otimes\dpi\bigr)^T\vc_j(t)\otimes\dpj\\
% 	 &= \sumi\sumj \bigl(\dpi\otimes\vc_i(t)\bigr)\vc_j(t)\otimes\dpj\\
% 	 &=\sumi\sumj (\vc_i(t)\cdot \vc_j(t))\dpi\otimes\dpj,\\
	\vF^T\vF\va\va^T &= \vF^T\left(\sum_j F_{ij}a_j\right)_i\va^T = \left(\sum_{i,j} F_{ik}F_{ij}a_j\right)_k\va^T
		= \left(\sum_{i,j} F_{ik}F_{ij}a_ja_l\right)_{kl}\\
	\Rightarrow \tr{(\vF^T\vF\va\va^T)} &= \sum_{i,j,k} F_{ik}F_{ij}a_ja_k = \sum_i \sum_jF_{ij}a_j \sum_k F_{ik}a_k = \sum_i (\vF\va)_i^2\\
	\Rightarrow \sqrt{\tr{(\vF^T\vF\va\va^T)}} &= \no{\vF\va}\\
	\la(X,t)^2 &=I_4(\vC(X,t),\va_0(X)) = \vC(X,t):(\va_0(X)\otimes \va_0(X))\\
			&=  \vF(X,t)^T\vF(X,t) : (\va_0(X)\otimes \va_0(X))\\
			&= \tr{\left(\vF(X,t)^T\vF(X,t)\va_0(X)\va_0(X)^T\right)}\\
% 			&=  \sumi\sumj (\vc_i(t)\cdot \vc_j(t))\bigl(\dpi\otimes\dpj\bigr) : (\va_0(X)\otimes \va_0(X))\\
% 			&=  \sumi\sumj (\vc_i(t)\cdot \vc_j(t))\left(\dpi\cdot \va_0(X)\right)(\dpj\cdot \va_0(X)),\\
	\Rightarrow \la(X,t) &=
			\sqrt{\tr{\left(\vF(X,t)^T\vF(X,t)\va_0(X)\va_0(X)^T\right)}} = \no{\vF(X,t)\va_0(X)} 
\end{align*}
%\left(\sumi\sumj (\vc_i(t)\cdot \vc_j(t))\left(\dNimp\cdot \va_0(\pmp)\right)(\dNjmp\cdot \va_0(\pmp))\right)^\frac{1}{2},\\
% \begin{align*}
%     \vF(X,t)(\va_0(X)\otimes\va_0(X)) &= \sumi (\vc_i(t)\otimes \dpi)(\va_0(X)\otimes\va_0(X))\\
%     &= \sumi (\vc_i(t)\otimes \va_0(X))(\dpi\cdot\va_0(X)),\\
%     \Rightarrow \vF(\pmp,t)(\va_0(\pmp)\otimes\va_0(\pmp)) 
%     	&= \sumi (\vc_i(t)\otimes \va_0(\pmp))(\dNimp\cdot\va_0(\pmp)).\\
% \end{align*}
For later maybe:
\begin{align*}
	\d{\la}{t}(X,t) &= \d{}{t}\no{\vF(X,t)\va_0(X)} = \frac{(\vF(X,t)\va_0(X))^T}{\no{\vF(X,t)\va_0(X)}}\d{}{t}\vF(X,t)\va_0(X)\\
	  &= \frac{1}{\la(X,t)}\va_0(X)^T\vF(X,t)^T\vF'(X,t)\va_0(X)
\end{align*}
% \begin{align*}
% 	\d{\la}{t}(X,t) &= \frac{1}{2\la(X,t)} \suml{i,j}{N} (\vc'_i(t)\cdot \vc_j(t) + \vc_i(t)\cdot \vc'_j(t))\left(\dpi\cdot a_0(X)\right)(\dpj\cdot a_0(X))\\
% 		&= \frac{1}{\la(X,t)}\suml{i,j}{N} (\vc_i(t)\cdot \vc'_j(t))\left(\dpi\cdot a_0(X)\right)(\dpj\cdot a_0(X))
% \end{align*}

%     \vF(X,t)\vF(X,t)^T\vF(X,t) &= \sumk\vc_k(t)\otimes\dpk\sumi\sumj (\vc_i(t)\cdot \vc_j(t))\dpi\otimes\dpj\nonumber\\
%     	&= \suml{i,j,k}{N}(\vc_i(t)\cdot \vc_j(t))(\vc_k(t)\otimes\dpk)\dpi\otimes\dpj\nonumber\\
%     	&= \suml{i,j,k}{N}(\vc_i(t)\cdot \vc_j(t))(\dpk\cdot\dpi)(\vc_k(t)\otimes\dpj)\nonumber\\

% 	\vF(\pmp,t)\vC(\pmp,t) &= \vF(\pmp,t)\vF(\pmp,t)^T\vF(\pmp,t)\\
%     	&= \suml{i,j,k}{N}(\vc_i(t)\cdot \vc_j(t))(\divergence\varphi_k(\pmp)\cdot\divergence\varphi_i(\pmp))(\vc_k(t)\otimes\divergence\varphi_j(\pmp))\\
	%&= \suml{i,j,q\in E_m}{}(\vc_i(t)\cdot \vc_j(t))(\nabla N_{l(q,m)}(X_p)\cdot\nabla N_{l(i,m)}(X_p))(\vc_q(t)\otimes\nabla N_{l(j,m)}(X_p))\\
	%&= \suml{i,j\in E_m}{} (\vc_i(t)\cdot \vc_j(t))\left(\nabla N_{l(i,m)}(X_p)\cdot a_0(\pmp)\right)(\nabla N_{l(j,m)}(X_p))\cdot a_0(\pmp))\nonumber
	%&= \suml{i\in E_m}{} (\vc_i(t)\otimes \va_0(\pmp))(\nabla N_{l(i,m)}(X_p)\cdot\va_0(\pmp))\nonumber

	

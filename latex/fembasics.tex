We specify a master hexahedron/volume/element $\Om = [-1,1]^3$ with corners
\begin{align*}
	\vX_{master} := \m{-1 & 1 & -1 & 1 & -1 & 1 & -1 &1\\-1 & -1 & 1 & 1 & -1 & -1 & 1 &1\\-1 & -1 & -1 & -1 & 1 & 1 & 1 &1\\}.
\end{align*}
This will be the ``natural'' order when referring to corners of the master volume, and we will stick with that order also for basis function indexing and all derived quantities.
All integrals will be performed on this master element $\Om$, where we apply a Gauss Quadrature with
\begin{align}
	X_1,\ldots,X_G  &\in \Om &&\text{Gauss points},\\
	w_1,\ldots,w_G  &\in \R &&\text{Gauss weights},
\end{align}
so that
\begin{align}
	\intom f(X)dX \approx \sumgp w_pf(X_p)\label{def:gaussintapprox}
\end{align}
for any integrable $f(X)$.
\subsubsection{Master basis functions}
\paragraph{Linear master basis functions}
For $\H^1$ we define the $n_b=8$ linear basis functions
\begin{align*}
	N_i:\Om &\to [0,1]\\
	N_i(X)  &= (1+(-1)^{i}X_1)(1+(-1)^{\ceil{\frac{i}{2}}}X_2)(1+(-1)^{\ceil{\frac{i}{4}}}X_3),\quad i=1\ldots n_b
\end{align*}
which amounts to trilinear functions equal to one in each corner of the hexahedron.
The set of functions has the property
\begin{align}
	\suml{i=1}{n_b}N_i &\equiv 1 & \text{i.e.}\quad \suml{i=1}{n_b}N_i(X) &= 1 \fo X\in\Om,\label{eq:basisfunsum}
\end{align}
which will be needed later.
\paragraph{Quadratic master basis functions}
For quadratic basis functions we need more Degrees of Freedom (DoF) in order to be able to satisfy \eqref{eq:basisfunsum}.
Well-known are $n_b=20$ or $n_b=27$ DoFs using extra edge-midpoint and mid-face locations.
We will use $n_b = 20$ with extra DoFs on the $12$ edge-midpoints.
Those be easily computed by a transformation matrix $T\in\R^{20 \times 8}$: 
\begin{lstlisting}
i = [1 2  2  3  4  4  5  5 6  7  7 8  9  9 10 10 11 11 12 12 13 14 14 15 16 16 17 17 18 19 19 20];  
j = [1 1  2  2  1  3  2  4 3  3  4 4  1  5 2  6  3  7  4  8  5  5  6  6  5  7  6  8  7  7  8  8];
s = [1 .5 .5 1 .5 .5 .5 .5 1 .5 .5 1 .5 .5 .5 .5 .5 .5 .5 .5 1  .5 .5 1  .5 .5 .5 .5 1  .5 .5 1];
T = sparse(i,j,s,20,8);
\end{lstlisting}
Then the extended ``corner'' set for the master hexahedron is $\vX_{master}T' \in\R^{20\times 3}$.
Now, for $\H^1$ we define the quadratic basis functions
\begin{align*}
	N_i:\Om &\to [0,1]\\
	N_1(X)  &= \frac{1}{8}(1-X_1)(1-X_2)(1-X_3)(-X_1-X_2-X_3-2)\\ % C1
    N_2(X)  &= \frac{1}{4}(1-X_1^2)(1-X_2)(1-X_3)\\ % E2
    N_3(X)  &= \frac{1}{8}(1+X_1)(1-X_2)(1-X_3)(X_1-X_2-X_3-2)\\ % C3
    N_4(X) &= \frac{1}{4}(1-X_2^2)(1-X_1)(1-X_3)\\ % E4
    N_5(X) &= \frac{1}{4}(1-X_2^2)(1+X_1)(1-X_3)\\ % E5
    N_6(X) &= \frac{1}{8}(1-X_1)(1+X_2)(1-X_3)(-X_1+X_2-X_3-2)\\ % C6
    N_7(X) &= \frac{1}{4}(1-X_1^2)(1+X_2)(1-X_3)\\ % E7
    N_8(X) &= \frac{1}{8}(1+X_1)(1+X_2)(1-X_3)(X_1+X_2-X_3-2)\\ % C8
    N_9(X) &= \frac{1}{4}(1-X_3^2)(1-X_1)(1-X_2)\\ % E9    
    N_{10}(X) &= \frac{1}{4}(1-X_3^2)(1+X_1)(1-X_2)\\ % E10
    N_{11}(X) &= \frac{1}{4}(1-X_3^2)(1-X_1)(1+X_2)\\ % E11
    N_{12}(X) &= \frac{1}{4}(1-X_3^2)(1+X_1)(1+X_2)\\ % E12
    N_{13}(X) &= \frac{1}{8}(1-X_1)(1-X_2)(1+X_3)(-X_1-X_2+X_3-2)\\ % C13
    N_{14}(X) &= \frac{1}{4}(1-X_1^2)(1-X_2)(1+X_3)\\ % E14
    N_{15}(X) &= \frac{1}{8}(1+X_1)(1-X_2)(1+X_3)(X_1-X_2+X_3-2)\\ %C15
    N_{16}(X) &= \frac{1}{4}(1-X_2^2)(1-X_1)(1+X_3)\\ % E16
    N_{17}(X) &= \frac{1}{4}(1-X_2^2)(1+X_1)(1+X_3)\\ % E17
    N_{18}(X) &= \frac{1}{8}(1-X_1)(1+X_2)(1+X_3)(-X_1+X_2+X_3-2)\\ % C18
    N_{19}(X) &= \frac{1}{4}(1-X_1^2)(1+X_2)(1+X_3)\\ % E19
    N_{20}(X) &= \frac{1}{4}(1+X_1)(1+X_2)(1+X_3)(X_1+X_2+X_3-2)]\\
\end{align*}
which amounts to triquadratic functions equal to one in each corner and edge-midpoints of the hexahedron.

\subsubsection{Domain decomposition}
Let $\Or$ be our domain of interest, decomposed into $\Or = \Omega_1\cup \ldots \cup \Omega_M$, where
$\Omega_m = [\vx^m_1,\ldots,\vx^m_{n_b}]$ is a deformed cube specified by the $n_b$ points $\vx_i^m\in\R^3$.
Let $\Ns := \{\vx_1,\ldots\vx_N\}$ denote the set of $N$ distinct nodes in
$\{\vx_1^1, \ldots, \vx_{n_b}^1, \vx_1^2, \ldots, \vx_{n_b}^{M-1}, \vx_1^M, \ldots, \vx_{n_b}^M\}$, numbered by first occurrence.

Using the notation
\begin{align}
	\vN(X) &:= \m{N_1(X) & \ldots & N_{n_b}(X)}^T\in\R^{n_b\times 1}\\
	\vX^m &:= \m{\vx^m_1 \ldots \vx^m_{n_b}} \in\R^{3\times n_b}
\end{align}
 we can specify an \e{isogeometric mapping} or diffeomorphism
\begin{align}
	\Phi_m : \Om &\to \Omega_m\\
	X &\mapsto \vX^m\vN(X)
\end{align}
which satisfies $\Omega_m = \Phi_m(\Or)$ and has the Jacobian
\begin{align}
	J\Phi_m(X) &= \m{\grad \Phi_{m_1}(X)\\\grad \Phi_{m_2}(X)\\\grad \Phi_{m_3}(X)\\} 
	= \vX^m\m{\grad N_1(X)\\ \vdots\\\grad N_{n_b}(X)} = \vX^m\nabla\vN(X) \in \R^{3\times 3}.\label{def:refbasis_jacobian}
\end{align}

\subsubsection{Node basis functions}
With the volume index set
\begin{align}
	%E_m &:= E(\Omega_m) := \{i\in\{1\ldots N\}~|~ \vx_i\in\Omega_m\}, \quad m=1\ldots M,\\
	V_k &:= \bigl\{i\in\{1\ldots M\}~|~ \vx_k\in\Omega_i\bigr\},&&k=1\ldots N
	%\Nk &:= \{i\in\{1\ldots N\}~|~ \vx_i\in E_m, m\in V_k\} = \{i ~|~ \supp\varphi_i \cap \supp\varphi_k \neq\es\},
\end{align}
we now define basis functions $\varphi_k$ on each node $k$ via
\begin{align}
	\varphi_k(X) &:= \begin{cases}
		N_{l(k,m)}(\Phi^{-1}_m(X)), & X\in\Omega_m, m\in V_k,\\
		0 & \text{else},	
	\end{cases}\label{def:referencebasisfun}\\
	l(k,m) &:= \{i\in\{1\ldots n_b\}~|~ N_i(\Phi_m^{-1}(\vx_k))=1\},\quad m=1\ldots M.
\end{align}
Here $l(k,m)$ refers to the local corner index of the basis function on $\Omega_m$ that equals one at $\vx_k$.


For the gradient of $\varphi_k$ we thus obtain
\begin{align}
	\grad\varphi_k(X) &\re{def:referencebasisfun} \grad (N_{l(k,m)}(\Phi^{-1}_m(X)))
					  = \grad N_{l(k,m)}(\Phi^{-1}_m(X))J\Phi^{-1}_m(X)\nonumber\\
					  &= \grad N_{l(k,m)}(\Phi^{-1}_m(X))(J\Phi_m(X))^{-1}\label{eq:gradvarphi}
\end{align}
for $X\in\Omega_m, m\in V_k$ (zero otherwise).
 
\subsubsection{Discrete integrals with node basis functions}
We define the shorthands
\begin{align}
	\pmp &:= \Phi_m(X_p)\\
	\jmp &:= |\det J\Phi_m(X_p)| = |\det \vX^m\nabla\vN(X_p)|.\label{def:jacshorthand}\\
	\dNkmp &:= \nabla\varphi_k(\pmp) \re{eq:gradvarphi} \left(J\Phi_m(X_p)\right)^{-T}\nabla \Nkmp)\in\R^3\label{}
\end{align}

With the above, and the transformation theorem and the local support of the basis function $\varphi_k$ we now have for any integrable $f(X,t)$ that
\begin{align}
   \intl{\Or}{}f(X,t)\varphi_k(X)dX &= \suml{m=1}{M}\intl{\Omega_m}{}f(X,t)\varphi_k(X)dX = \sumvk\intl{\Omega_m}{}f(X,t)\varphi_k(X)dX\nonumber\\
    &= \sumvk\intl{\Phi_m(\Omega)}{}f(X,t)\varphi_k(X)dX\nonumber\\
	&= \sumvk\intl{\Om}{}f(\Phi_m(X),t)\varphi_k(\Phi_m(X))|\det J\Phi_m(X)|dX\nonumber\\
	& \re{def:gaussintapprox} \sumvk\sumgp w_pf(\pmp,t)\varphi_k(\pmp)|\det J\Phi_m(X_p)|\nonumber\\
	& \re{def:referencebasisfun} \sumvk\sumgp w_pf(\pmp,t)N_{l(k,m)}(\Phi_m^{-1}(\pmp))|\det J\Phi_m(X_p)|\nonumber\\
	& \re{def:jacshorthand} \sumvk\sumgp w_pf(\pmp,t)\Nkmp\jmp\label{eq:fphidx}
\end{align}

Similarly we obtain for the gradient that
\begin{align}
	\intl{\Or}{}f(X,t)\nabla\varphi_k(X)dX &= \sumvk\sumgp w_pf(\pmp,t)\dNkmp\jmp\label{eq:fgradphidx}
\end{align}